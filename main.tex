\documentclass{article}
\usepackage{lmodern}
\usepackage[T1]{fontenc}
\usepackage{indentfirst}
\usepackage{color}
\usepackage{minted}
\usepackage{hyperref}
\usepackage{amssymb} 
\usepackage{algorithm}
\usepackage{pygmentex}
\usepackage{amsmath}
\usepackage{algpseudocode}
\usepackage[many]{tcolorbox}
\usepackage{multicol}
\usepackage{setspace}
\usepackage{paracol}
\usepackage{algpseudocode}
\usepackage{makecell}
\usepackage{silence,lmodern}
\usepackage[margin=1in]{geometry}

\author{Gustavo Ribeiro Bôdi e Pedro Alfeu}
\title{\vspace{-2.2cm}Relatório Trabalho 2 Máquinas de Turing e Complexidade}


\makeatletter
\newenvironment{breakablealgorithm}
  {% \begin{breakablealgorithm}
   \begin{center}
     \refstepcounter{algorithm}% New algorithm
     \hrule height.8pt depth0pt \kern2pt% \@fs@pre for \@fs@ruled
     \renewcommand{\caption}[2][\relax]{% Make a new \caption
       {\raggedright\textbf{\ALG@name~\thealgorithm} ##2\par}%
       \ifx\relax##1\relax % #1 is \relax
         \addcontentsline{loa}{algorithm}{\protect\numberline{\thealgorithm}##2}%
       \else % #1 is not \relax
         \addcontentsline{loa}{algorithm}{\protect\numberline{\thealgorithm}##1}%
       \fi
       \kern2pt\hrule\kern2pt
     }
  }{% \end{breakablealgorithm}
     \kern2pt\hrule\relax% \@fs@post for \@fs@ruled
   \end{center}
  }
\makeatother


\renewcommand\theadalign{bc}
\renewcommand\theadfont{\bfseries}
\renewcommand\theadgape{\Gape[4pt]}
\renewcommand\cellgape{\Gape[8pt]}
\newtcolorbox{boxK} {
  sharpish corners,
  boxrule = 0pt,
  toprule = 4.5pt,
  enhanced,
  fuzzy shadow = {0pt}{-2pt}{-0.5pt}{0.5pt}{black!35} % {xshift}{yshift}{offset}{step}{options} 
}

\algnewcommand\algorithmicforeach{\textbf{for each}}
\algdef{S}[FOR]{ForEach}[1]{\algorithmicforeach\ #1\ \algorithmicdo}

\algrenewcommand\algorithmicrequire{\textbf{Input:}}
\algrenewcommand\algorithmicensure{\textbf{Output:}}

\begin{document}
\maketitle
\section{Análise de Complexidade}

Para a análise de complexidade dos algoritmos, descreveu-se os algoritmos de
acordo com as tabelas, e á direita de cada linha, encontra-se a complexidade de
cada operação. Após uma breve descrição dos procedimentos, fez-se a soma, dos
valores, as devidas simplificações e apresentou-se as complexidades finais.

Nas descrições, quando uma linguagem $L$ detêm, por exemplo $a$'s e $b$'s,
para identificar a quantidade de $a$'s, convencionou-se por: $"a"$.

\subsection{Questão(a)}
Na primeira questão tem-se a seguinte linguagem: $$L=\{a^ib^jc^k \mid i,j,k \in \mathbb{N}\ e\ i\times j = k\}$$

\subsubsection{Fita Única Original}

O algoritmo em alto nível em uma máquina de fita única que resolve a linguagem é dado por:

\begin{algorithm} 
  \caption{Algoritmo que resolve a linguagem da questão 1a}
\begin{algorithmic}[1]
  \Require{$w$}
  \Ensure{boolean}
\ForEach {$a \in w $}
\Comment {$\mathcal{O}(a)$}
\State $a \gets A$
\Comment {$\mathcal{O}(c_1)$}
\State Passar pelos $a$
\Comment {$\mathcal{O}("a" - "A")$}
\State Passar pelos $B$
\Comment {$\mathcal{O}("B")$}
\ForEach{$b \in w$}
\Comment {$\mathcal{O}("b")$}
  \State $b \gets B$
  \Comment {$\mathcal{O}(c_2)$}
  \State Passar pelos $b$
  \Comment {$\mathcal{O}("b" - "B")$}
  \State Passar pelos $C$
  \Comment {$\mathcal{O}("c" - "C")$}
  \State $c \gets C$
  \Comment {$\mathcal{O}(c_3)$}
  \State Voltar todos os $C$
  \Comment {$\mathcal{O}("C")$}
  \State Voltar todos os $b$
  \Comment {$\mathcal{O}("b")$}
\EndFor
\ForEach{$B \in w$}
  \Comment {$\mathcal{O}("B")$}
\State $B \gets b$
\Comment {$\mathcal{O}(c_4)$}
\EndFor
\EndFor
\ForEach{$c \in w$}
\Comment {$\mathcal{O}("C" - c)$}
\If {Existir $c$}
  \Comment {$\mathcal{O}(c_5)$}
  \State \Return false
  \Comment {$\mathcal{O}(c_6)$}
  \EndIf
\EndFor
\State \Return true
  \Comment {$\mathcal{O}(c_7)$}
\end{algorithmic}
\end{algorithm}

Basicamente, para cada a, marca-se cada b e cada c, depois limpa-se as
marcações de B e marca-se para cada B cada C de novo.

Resolvendo assim a complexidade do algoritmo da questão têm-se:

Da linha 1 até a linha 14:
\begin{align}
  \begin{split}
    \mathcal{O}(n) = &{}\\
                     &\mathcal{O}("a") \cdot (\mathcal{O}(c_1) + \mathcal{O}("a"-"A") + \mathcal{O}("B") + \\
                     &\mathcal{O}("b") \cdot (\mathcal{O}(c_2) + \mathcal{O}("b"-"B") + \mathcal{O}("c"-"C") + \mathcal{O}(c_3)) + \\
                     &\mathcal{O}("B") \cdot \mathcal{O}(c_4)) \\
\end{split}
\end{align}

Da linha 15 até a linha 20:
$$\mathcal{O}("C"-"c") \cdot \mathcal{O}(c_5) + \mathcal{O}(c_6) + \mathcal{O}(c_7)$$

Considerando todos os $c_x$ como sendo $c$ qualquer e juntando ambas as expressões:
\begin{align}
  \begin{split}
    \mathcal{O}(n) = &{}\\
                     &\mathcal{O}("a") \cdot (\mathcal{O}(c) + \mathcal{O}("a"-"A") + \mathcal{O}("B") + \\
                     &\mathcal{O}("b") \cdot (\mathcal{O}(c) + \mathcal{O}("b"-"B") + \mathcal{O}("c"-"C") + \mathcal{O}(c)) + \\
                     &\mathcal{O}("B") \cdot \mathcal{O}(c)) + \\
                     &\mathcal{O}("C"-"c") \cdot \mathcal{O}(c) + \mathcal{O}(c) + \mathcal{O}(c) \\
  \end{split}
\end{align}
Considerando a quantidade de $"a" - "A"$ como sendo $"a"$ e o equivalente para $b, c$. E ainda que $"A"$ no pior caso é $"a"$:
\begin{align}
  \begin{split}
    \mathcal{O}(n) = &{}\\
                     &\mathcal{O}("a") \cdot (\mathcal{O}(c) + \mathcal{O}("a") + \mathcal{O}("b") + \\
                     &\mathcal{O}("b") \cdot (\mathcal{O}(c) + \mathcal{O}("b") + \mathcal{O}("c") + \mathcal{O}(c)) + \\
                     &\mathcal{O}("b") \cdot \mathcal{O}(c)) + \\
                     &\mathcal{O}("c") \cdot \mathcal{O}(c) + \mathcal{O}(c) + \mathcal{O}(c) \\
  \end{split}
\end{align}
Simplificando, temos que:
\begin{align}
  \begin{split}
    \mathcal{O}(n) = &{}\\
                     &\mathcal{O}("a") \cdot (c + \mathcal{O}("a") + \mathcal{O}("b") + \\
                     &\mathcal{O}("b") \cdot (c + \mathcal{O}("b") + \mathcal{O}("c") + c) + \\
                     &\mathcal{O}("b") \cdot c) + \\
                     &\mathcal{O}("c") \cdot c + c + c \\
    \mathcal{O}(n) = &{}\\
                     &\mathcal{O}("a" \cdot c + "a"^2 + "a" \cdot "b" + "a"\cdot "b" \cdot c + \\
                     &"a"\cdot"b"^2 + "a" \cdot "b" \cdot "c" + "a" \cdot "b" \cdot c + "b" \cdot c) + \\
                     &\mathcal{O}("c") \cdot 3c
  \end{split}
\end{align}
Considerando a quantidade de $a$ $"a"$, $b$ $"b"$ e $c$ $"c"$ como sendo $n$:
\begin{align}
  \begin{split}
    \mathcal{O}(n) = &{}\\
                     &\mathcal{O}(n \cdot c + n^2 + n \cdot n + n\cdot n \cdot c + \\
                     &n\cdot n^2 + n \cdot n \cdot n + n \cdot n \cdot c + n \cdot c) +\\
                     &\mathcal{O}(n) \cdot 3c \\
    \mathcal{O}(n) = &{}\\
                     &\mathcal{O}(n \cdot c + n^2 + n^2 + cn^2 + \\
                     &n^3 + n^3 + cn^2 + cn) + \\
                     &\mathcal{O}(n) \cdot 3c \\
  \end{split}
\end{align}

\begin{align}
  \begin{split}
    \mathcal{O}(n) = \mathcal{O}(2n^3 + 2n^2 + 2cn^2 + 4cn)
  \end{split}
\end{align}

Considerando somente o termo mais significativo:
$$\mathcal{O}(n^3)$$
\pagebreak

\subsubsection{Multifitas Otimizada}

Para a máquina multifitas o algoritmo será uma classe de complexidade melhor.
Aqui, ao fazer uso de três fitas, a multiplicação será dada em tempo linear à
entrada. O procedimento será copiar "a" para uma segunda fita e "b" para uma
terceira, a partir disso ao encontrar um c mover a cabeçote$_2$ para o primeiro
$a$ e mover o cabeçote$_3$ para cada $c$ encontrado. Quando o cabeçote$_3$
encontrar um caracter vazio, basta mover o cabeçote$_2$ para o próximo $a$ e
continuar até terminar todos os $c$.

A descrição alto nível do algoritmo é dada por:

\begin{algorithm} 
  \caption{Algoritmo que resolve a linguagem da questão 1a em tempo linear}
\begin{algorithmic}[1]
  \Require{$w$}
  \Ensure{boolean}
  \ForEach{$a \in w$}
  \Comment $\mathcal{O}("a")$
    \State Copiar $a$ para a fita$_2$
  \Comment $\mathcal{O}(c_1)$
  \EndFor
  \ForEach{$b \in w$}
  \Comment $\mathcal{O}("b")$
    \State Copiar $b$ para a fita$_3$
  \Comment $\mathcal{O}(c_2)$
  \EndFor
  \ForEach{$c \in w$}
  \Comment $\mathcal{O}("c")$
  \If{cabeçote$_3$ == vazio}
  \State Passar para o próximo $a$
  \Comment $\mathcal{O}(c_4)$
  \If{cabeçote$_2$, cabeçote$_3$ == vazio}
    \State \Return true
  \EndIf
  \State Inverter a direção de iteração da fita$_3$
  \Comment $\mathcal{O}(c_5)$
  \EndIf
  \State Passar para o próximo caracter na fita$_2$
  \Comment $\mathcal{O}(c_6)$
  \EndFor
\end{algorithmic}
\end{algorithm}

O cálculo de complexidade será dado por:

\begin{align}
  \begin{split}
    \mathcal{O}(n) = \mathcal{O}("a") \cdot c_1 + \mathcal{O}("b") \cdot c_2 + \mathcal{O}("c") \cdot c_6 + \frac{\mathcal{O}("c") \cdot (c_4 + c_5)}{"a"}  \\
\end{split}
\end{align}

Vale notar que o condicional da linha 7, para o pior caso, acontecerá para cada a existente na sequência de caracteres.
Simplificando, considerando todas as constantes $c_x$ como $c$:

\begin{align}
  \begin{split}
    \mathcal{O}(n) = \mathcal{O}("a" \cdot c + "b" \cdot c +"c" \cdot c + \frac{"c" \cdot 2c }{"a"})  \\
\end{split}
\end{align}

Como temos que a entrada corresponde a $n = "a" + "b" + "c"$:
\begin{align}
  \begin{split}
    \mathcal{O}(n) = \mathcal{O}(c \cdot (n + \frac{"c" \cdot 2 }{"a"})  \\
\end{split}
\end{align}

Considerando somente o termo de maior importância, teremos:
\begin{align}
  \begin{split}
    \mathcal{O}(n)
\end{split}
\end{align}

\pagebreak

\subsection{Questão(b)}
Na segunda questão têm-se a seguinte linguagem: $$L=\{a^nb^mc^pd^q\mid n<m,m \neq p + q\ \text{e}\ p > q\}$$

\subsubsection{Fita Única original}

O algoritmo de alto nível que resolve a linguagem é dado por:

\begin{breakablealgorithm} 
  \caption{Algoritmo que resolve a linguagem da questão 1b}
\begin{algorithmic}[1]
  \Require{$w$}
  \Ensure{boolean}
\ForEach {$a \in w $}
\Comment{Lidando aqui com a restrição $n<m$ $\mathcal{O}("a")$}
\State $a \gets A$
\Comment $\mathcal{O}(c_1)$
\State passar por todos os $a$ e $B$
\Comment $\mathcal{O}("a" - 1 + "B")$
\State $b \gets B$
\Comment $\mathcal{O}(c_2)$
\State voltar até encontrar um $A$
\Comment $\mathcal{O}("a" + "B")$
\EndFor
\If {$b \not \in w$}
\Comment $\mathcal{O}("B")$
\State \Return false
\EndIf
\ForEach {$B, A \in w$}
\Comment $\mathcal{O}("B" + "A")$
\State $B \gets b$
\Comment $\mathcal{O}(c_3)$
\State $A \gets a$
\Comment $\mathcal{O}(c_4)$
\EndFor
\ForEach {$b \in w $}
\Comment{Lidando aqui com a restrição $m\neq p+q$ $\mathcal{O}("b")$}
\State $b \gets B$
\Comment $\mathcal{O}(c_5)$
\State passar por todos os $b$, $C$ e $D$
\Comment $\mathcal{O}("b" + "c" + "D")$
\If {cabeçote = $c$}
  \Comment $\mathcal{O}(c_6)$
  \State $c \gets C$
  \Comment $\mathcal{O}(c_7)$
  \ElsIf{cabeçote = $d$}
  \Comment $\mathcal{O}(c_8)$
  \State $d \gets D$
  \Comment $\mathcal{O}(c_9)$
\EndIf
\State voltar até encontrar um $B$
  \Comment $\mathcal{O}("C" + "b")$
\EndFor
\If {$c \not \in w\ \text{and}\ d \not \in w$}
  \Comment $\mathcal{O}("c" + "d")$
\State \Return false
\Comment $\mathcal{O}(c_{10})$
\EndIf
\ForEach {$B, C, D \in w$}
\Comment $\mathcal{O}("B" + "C" + "D")$
\State $B \gets b$
\Comment $\mathcal{O}(c_{11})$
\State $C \gets c$
\Comment $\mathcal{O}(c_{12})$
\State $D \gets d$
\Comment $\mathcal{O}(c_{13})$
\EndFor
\ForEach{$c \in w$}
\Comment{Lidando aqui com a restrição $p>q$ $\mathcal{O}("c")$}
\State $c \gets C$
\Comment $\mathcal{O}(c_{14})$
\State passar por todos os $D$ e $c$
\Comment $\mathcal{O}("D" + "c")$
\State $d \gets D$
\Comment $\mathcal{O}(c_{15})$
\State voltar até encontrar C
\Comment $\mathcal{O}("D" + "c")$
\If {cabeçote = vazio}
\Comment $\mathcal{O}(c_{16})$
\State \Return true
\Comment $\mathcal{O}(c_{17})$
\EndIf
\EndFor
\State \Return false
\Comment $\mathcal{O}(c_{18})$
\end{algorithmic}
\end{breakablealgorithm}

Percebe-se aqui que o algoritmo está efetivamente dividido em três partes, cada
uma delas lida com cada uma das limitações impostas pela linguagem.

\subsubsection{ Multifitas Modificado }
Para o multifitas da questão 1b, de novo foi possível resolver a linguagem
fazendo uso de um atômato que computa em tempo linear. Para tal, foi necessário
usar uma fita para cada uma das comparações da linguagem, totalizando 4 fitas:
3 comparações + uma fita de entrada.

Esta solução envolveu movimentar o cabeçote de cada fita de acordo com a
leitura da entrada e somente transicionar para o próximo estado caso fizesse
sentido de acordo com as comparações da linguagem.

O algoritmo de alto nível que descreve a nova máquina é descrito como:

\begin{breakablealgorithm} 
  \caption{Algoritmo que resolve a linguagem da questão 1b em tempo linear}
\begin{algorithmic}[1]
  \Require{$w$}
  \Ensure{boolean}
  \ForEach{$a \in w$}
  \Comment $\mathcal{O}("a")$
  \State Escrever a na fita$_2$
  \Comment $\mathcal{O}(c_1)$
  \EndFor
  \State Escrever símbolo de inicio na fita$_3$
  \Comment $\mathcal{O}(c_2)$
  \ForEach{$b \in w$}
  \Comment $\mathcal{O}("b")$
  \State Retroceder a fita$_2$
  \Comment $\mathcal{O}(c_3)$
  \State Avançar na fita$_3$
  \Comment $\mathcal{O}(c_4)$
  \EndFor
  \ForEach{$c \in w$}
  \Comment $\mathcal{O}("c")$
  \State Retroceder a fita$_3$
  \Comment $\mathcal{O}(c_5)$
  \State Escrever c na fita$_4$ e avança
  \Comment $\mathcal{O}(c_6)$
  \EndFor
  \ForEach{$d \in w$}
  \Comment $\mathcal{O}("d")$
  \State Retroceder a fita$_3$
  \Comment $\mathcal{O}(c_7)$
  \State Retroceder a fita$_4$
  \Comment $\mathcal{O}(c_8)$
  \EndFor
  \If{cabeçote$_1$, cabeçote$_2$, cabeçote$_3$ = vazio e cabeçote$_4$ = $c$}
  \Comment $\mathcal{O}(c_9)$
  \State \Return true
  \Comment $\mathcal{O}(c_{10})$
  \Else
  \State \Return false
  \Comment $\mathcal{O}(c_{10})$
  \EndIf
\end{algorithmic}
\end{breakablealgorithm}

O cálculo de complexidade será dado por:
\begin{align}
  \begin{split}
    \mathcal{O}(n) = \mathcal{O}("a") \cdot c_1 + \mathcal{O}(c_2) + \mathcal{O}("b") \cdot (c_3 + c_4) + \mathcal{O}("c") \cdot (c_5 + c_6) + \mathcal{O}("d") \cdot (c_7 + c_8) + c_9 + c_{10} \\
\end{split}
\end{align}

Simplificando, considerando todas as constantes $c_x$ como $c$:
\begin{align}
  \begin{split}
    \mathcal{O}(n) = \mathcal{O}("a" \cdot c + c + "b" \cdot 2c + "c" \cdot 2c + "d" \cdot 2c + 2c) \\
    \mathcal{O}(n) = \mathcal{O}(c"a" + 2c"b" + 2c"c" + 2c"d" + 3c) \\
    \mathcal{O}(n) = \mathcal{O}(c \cdot ("a" + 2"b" + 2"c" + 2"d" + 3)) \\
\end{split}
\end{align}

Como as ações dentro dos loops são executadas em "paralelo", ou em fitas
diferentes ao mesmo tempo, pode-se remover o fator multiplicativo 2 das
constantes. Além disso, como temos que a entrada corresponde a $n = "a" + "b" + "c" + "d"$:
\begin{align}
  \begin{split}
    \mathcal{O}(n) = \mathcal{O}(c \cdot (n + 3)) \\
\end{split}
\end{align}

Considerando somente o termo de maior importância, teremos:
\begin{align}
  \begin{split}
    \mathcal{O}(cn) \\
    \mathcal{O}(n)
\end{split}
\end{align}

\end{document}
